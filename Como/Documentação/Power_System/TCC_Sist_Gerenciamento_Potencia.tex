% ------------------------------------------------------------------------
% ------------------------------------------------------------------------
% abnTeXf2: Modelo de Trabalho Academico (tese de doutorado, dissertacao de
% mestrado e trabalhos monograficos em geral) em conformidade com 
% ABNT NBR 14724:2011: Informacao e documentacao - Trabalhos academicos -
% Apresentacao
% ------------------------------------------------------------------------
% ------------------------------------------------------------------------

\documentclass[
	% -- opções da classe memoir --
	12pt,				% tamanho da fonte
	openright,			% capítulos começam em pág ímpar (insere página vazia caso preciso)
	oneside,			% para impressão em verso e anverso. Oposto a oneside
	a4paper,			% tamanho do papel. 
	% -- opções da classe abntex2 --
	%chapter=TITLE,		% títulos de capítulos convertidos em letras maiúsculas
	%section=TITLE,		% títulos de seções convertidos em letras maiúsculas
	%subsection=TITLE,	% títulos de subseções convertidos em letras maiúsculas
	%subsubsection=TITLE,% títulos de subsubseções convertidos em letras maiúsculas
	% -- opções do pacote babel --
	english,			% idioma adicional para hifenização
	french,				% idioma adicional para hifenização
	spanish,			% idioma adicional para hifenização
	brazil,				% o último idioma é o principal do documento
	oldfontcommands
	]{abntex2}

% ---
% Pacotes básicos 
% ---
\usepackage{lmodern}			% Usa a fonte Latin Modern			
\usepackage[T1]{fontenc}		% Selecao de codigos de fonte.
\usepackage[utf8]{inputenc}		% Codificacao do documento (conversão automática dos acentos)
\usepackage{lastpage}			% Usado pela Ficha catalográfica
\usepackage{indentfirst}		% Indenta o primeiro parágrafo de cada seção.
\usepackage{color}				% Controle das cores
\usepackage{graphicx}			% Inclusão de gráficos
\usepackage{microtype} 			% para melhorias de justificação
\usepackage{eurosym}			% habilita símbolo do Euro
% ---
		
% ---
% Pacotes adicionais, usados apenas no âmbito do Modelo Canônico do abnteX2
% ---
\usepackage{lipsum}				% para geração de dummy text
% ---

% ---
% Pacotes de citações
% ---
\usepackage[brazilian,hyperpageref]{backref}	 % Paginas com as citações na bibl
\usepackage[num]{abntex2cite}	% Citações padrão ABNT


% --- 
% CONFIGURAÇÕES DE PACOTES
% --- 

% ---
% Configurações do pacote backref
% Usado sem a opção hyperpageref de backref
\renewcommand{\backrefpagesname}{Citado na(s) página(s):~}
% Texto padrão antes do número das páginas
\renewcommand{\backref}{}
% Define os textos da citação
\renewcommand*{\backrefalt}[4]{
	\ifcase #1 %
		Nenhuma citação no texto.%
	\or
		Citado na página #2.%
	\else
		Citado #1 vezes nas páginas #2.%
	\fi}%
% ---

% ---
% Informações de dados para CAPA e FOLHA DE ROSTO
% ---
\titulo{SISTEMA DE GERENCIAMENTO DE ENERGIA PARA CUBESAT}
\autor{ARNALDO ALVES VIANA JÚNIOR\\
OTÁVIO MOREIRA PETITO\\
TIAGO AUGUSTO ORCAJO DEMAY CORDEIRO}
\local{SÃO CAETANO DO SUL}
\data{2015}
\orientador{Prof. Me. Alessandro de Oliveira Santos}
%\coorientador{Professor}
\instituicao{Escola de Engenharia Mauá}
%tipotrabalho{Tese (Trabalho de Conclusão de Curso)}
% O preambulo deve conter o tipo do trabalho, o objetivo, 
% o nome da instituição e a área de concentração 
\preambulo{Trabalho de Conclusão de Curso apresentado à Escola de Engenharia Mauá do Centro Universitário do Instituto Mauá de Tecnologia, como parte dos requisitos necessários à obtenção do grau de bacharel em Engenharia na habilitação Engenharia Eletrônica. Área de concentração: Engenharia Elétrica}
% ---


% ---
% Configurações de aparência do PDF final

% alterando o aspecto da cor azul
\definecolor{blue}{RGB}{41,5,195}

% informações do PDF
\makeatletter
\hypersetup{
     	%pagebackref=true,
		pdftitle={\@title}, 
		pdfauthor={\@author},
    	pdfsubject={\imprimirpreambulo},
	    pdfcreator={LaTeX with abnTeX2},
		pdfkeywords={abnt}{latex}{abntex}{abntex2}{trabalho acadêmico}, 
		colorlinks=true,       		% false: boxed links; true: colored links
    	linkcolor=black,          	% color of internal links
    	citecolor=black,        		% color of links to bibliography
    	filecolor=black,      		% color of file links
		urlcolor=black,
		bookmarksdepth=4 
}
\makeatother
% --- 

% --- 
% Espaçamentos entre linhas e parágrafos 
% --- 

% O tamanho do parágrafo é dado por:
\setlength{\parindent}{1.25cm}

% Controle do espaçamento entre um parágrafo e outro:
\setlength{\parskip}{0.2cm}  % tente também \onelineskip

% ---
% compila o indice
% ---
\makeindex
% ---

% ----
% Início do documento
% ----
\begin{document}

% Retira espaço extra obsoleto entre as frases.
\frenchspacing 
% ----------------------------------------------------------
% ELEMENTOS PRÉ-TEXTUAIS
% ----------------------------------------------------------
% \pretextual

% ---
% Capa
% ---
\imprimircapa
% ---

% ---
% Folha de rosto
% (o * indica que haverá a ficha bibliográfica)
% ---
\imprimirfolhaderosto*
% ---

% ---
% Inserir a ficha bibliografica
% ---

% Isto é um exemplo de Ficha Catalográfica, ou ``Dados internacionais de
% catalogação-na-publicação''. Você pode utilizar este modelo como referência. 
% Porém, provavelmente a biblioteca da sua universidade lhe fornecerá um PDF
% com a ficha catalográfica definitiva após a defesa do trabalho. Quando estiver
% com o documento, salve-o como PDF no diretório do seu projeto e substitua todo
% o conteúdo de implementação deste arquivo pelo comando abaixo:
%
% \begin{fichacatalografica}
%     \includepdf{fig_ficha_catalografica.pdf}
% \end{fichacatalografica}
\begin{fichacatalografica}
	\vspace*{\fill}					% Posição vertical
	\hrule							% Linha horizontal
	\begin{center}					% Minipage Centralizado
	\begin{minipage}[c]{12.5cm}		% Largura
	
	Júnior, Arnaldo Alves Viana
	
	\hspace{0.5cm} \imprimirtitulo  / Arnaldo Alves Viana Júnior, Otávio Moreira Petito e Tiago Augusto Orcajo Demay. --
	\imprimirlocal, CEUN-EEM, \imprimirdata
	
	\hspace{0.5cm} \pageref{LastPage} p. : il. \\
	
	\hspace{0.5cm}
	\parbox[t]{\textwidth}{\imprimirtipotrabalho~--~\imprimirinstituicao,
	\imprimirlocal, \imprimirdata.}\\
	
	\hspace{0.5cm} \imprimirorientadorRotulo~\imprimirorientador\\
	
	\hspace{0.5cm}
		1. Gerenciamento de energia.
		2. CubeSat.
		I. Petito, Otávio Moreira.
		II. Cordeiro, Tiago Augusto Orcajo Demay
		III. Instituto Mauá de Tecnológia. Centro Universitário.
		IV. Título\\ 			
	
	\hspace{8.75cm} CDU \\
	
	\end{minipage}
	\end{center}
	\hrule
\end{fichacatalografica}
% ---

% ---
% Inserir errata
% ---

% ---

% ---
% Inserir folha de aprovação
% ---

% Isto é um exemplo de Folha de aprovação, elemento obrigatório da NBR
% 14724/2011 (seção 4.2.1.3). Você pode utilizar este modelo até a aprovação
% do trabalho. Após isso, substitua todo o conteúdo deste arquivo por uma
% imagem da página assinada pela banca com o comando abaixo:
%
% \includepdf{folhadeaprovacao_final.pdf}
%
\begin{folhadeaprovacao}

  \begin{center}
    {\ABNTEXchapterfont\large\imprimirautor}

    \vspace*{\fill}\vspace*{\fill}
    \begin{center}
      \ABNTEXchapterfont\bfseries\Large\imprimirtitulo
    \end{center}
    \vspace*{\fill}
    
    \hspace{.45\textwidth}
    \begin{minipage}{.5\textwidth}
    \end{minipage}%
    \vspace*{\fill}
   \end{center}
        
Trabalho de Conclusão de Curso aprovado em \_\_\_ de \_\_\_\_\_\_\_\_\_\_\_\_\_ de 2015, pela banca examinadora composta por: 

   \assinatura{\textbf{\imprimirorientador} \\ Orientador} 
   \assinatura{\textbf{Professor} \\ Convidado 1}
   \assinatura{\textbf{Professor} \\ Convidado 2}
   %\assinatura{\textbf{Professor} \\ Convidado 3}
   %\assinatura{\textbf{Professor} \\ Convidado 4}
      
   \begin{center}
    \vspace*{0.5cm}
    {\large\imprimirlocal}
    \par
    {\large\imprimirdata}
    \vspace*{1cm}
  \end{center}
  
\end{folhadeaprovacao}
% ---

% ---
% Dedicatória
% ---
% ---

% ---
% Agradecimentos
% ---
\begin{agradecimentos}

A Escola de Engenharia Mauá por fornecer toda a gama de conhecimento e estrutura para um melhor aprendizado.

Ao {\imprimirorientador}  pela assessoria prestada quanto ao desenvolvimento do tema.

Ao Prof. Rafael Corsi por todo empenho dedicado auxiliando o projeto de distintas maneiras.

E aos nossos pais, amigos e namoradas que apesar de todas as dificuldades sempre nos suportaram para o melhor desenvolvimento do projeto.

\end{agradecimentos}
% ---

% ---
% Epígrafe
% ---
\begin{epigrafe}
    \vspace*{\fill}
	\begin{flushright}
		\textit{To infinity... and beyond!\\(Buzz Lightyear)}
		%\textit{Ao infinito... e além!\\(Buzz Lightyear}
		%\textit{O maior bem do homem é uma mente inquieta.\\(Isaac Asimov)}
	\end{flushright}
\end{epigrafe}
% ---

% ---
% RESUMOS
% ---

% resumo em português
\setlength{\absparsep}{18pt} % ajusta o espaçamento dos parágrafos do resumo
\begin{resumo}
 
	Nesse trabalho é abordado o desenvolvimento de um sistema de potência (\textit{EPS}) capaz de suprir toda a demanda energética dos subsistemas de controle de atitude, comunicação, processamento de dados e de carga útil do projeto \textit{CubeSat} da Escola de Engenharia Mauá. O sistema de potência desenvolvido é responsável pela geração (30W), distribuição e controle de todo fluxo de energia do \textit{CubeSat} Mauá. A energia gerada pelas fotocélulas aeroespaciais de alta eficiência (28\%), dotadas da tecnologia de tripla junção (GaInP/GaAs/Ge) é armazenada em baterias de Ion-Lithium. A distribuição da energia é realizada por três níveis de tensões estabilizados e regulados em 3V3, 5V e 12V, além de um nível não regulado fornecido diretamente das baterias. Em caso de falhas, um conjunto de fontes redundantes é capaz de assumir qualquer um dos níveis de tensões regulados. Todo o controle do sistema de potência é realizado por um microcontrolador, que coleta e analisa os dados, como temperatura, tensão e corrente, para determinar se a alimentação do sistema será proveniente das fontes principais ou das fontes redundantes. Através da rede CAN, o microcontrolador transmite as informações de telemetria para que a \textit{DPU} tome decisões mais complexas que envolvam todos os subsistemas do \textit{CubeSat}.

 \textbf{Palavras-chaves}: \textit{EPS}, sistema de potência, \textit{CubeSat}, gerenciamento de potência.
\end{resumo}

% resumo em inglês
\begin{resumo}[Abstract]
 \begin{otherlanguage*}{english}
   This is the english abstract.

   \vspace{\onelineskip}
 
   \noindent 
   \textbf{Key-words}: word1, word2, word3, word4, word5.
 \end{otherlanguage*}
\end{resumo}
% ---

% ---
% inserir lista de ilustrações
% ---
\pdfbookmark[0]{\listfigurename}{lof}
\listoffigures*
\cleardoublepage
% ---

% ---
% inserir lista de tabelas
% ---
\pdfbookmark[0]{\listtablename}{lot}
\listoftables*
\cleardoublepage
% ---

% ---
% inserir lista de abreviaturas e siglas
% ---
\begin{siglas}

  \item[\textit{GPS}] \textit{Global Positioning System}
  \item[NSEE-IMT] Núcleo de Sistemas Eletrônicos Embarcados do Instituto Mauá de Tecnologia
  \item[\textit{Cal Poly}] \textit{California Polytechnic State University}
  \item[\textit{NASA}] \textit{National Aeronautics and Space Administration}
  \item[\textit{LEO}] \textit{Low Earth Orbit}
  \item[\textit{PV}] \textit{Photovoltaic}
  \item[\textit{AM}] \textit{Air Mass}
  \item[\textit{WRC}] \textit{World Radiation Center}
  \item[T] Temperatura
  \item[\textit{SMD}] \textit{Superficial Monting Device}
  \item[ESR]
  \item[\textit{DF}] \textit{Dissipation Factor}
  \item[FEI] Faculdade de Engenharia Industrial
  \item[\textit{OSSI}] \textit{Open Source Satellite Initiative}
  \item[\textit{SEL}]\textit{Single Event Latchup}
  \item[\textit{SEE}]\textit{Single Event Effect}
  \item[\textit{SET}]\textit{Single Event Transient}
  \item[USP] Universidade de São Paulo
  
  
    
\end{siglas}
% ---

% ---
% inserir lista de símbolos
% ---
\begin{simbolos}
  \item[$ \Omega $] Unidade de medida ohm no Sistema Internacional de Unidades.
  \item[$ \mu $] Micro, equivalente a 10\textsuperscript{-6} no Sistema Internacional de Unidades.
%  \item[$ \zeta $] Letra grega minúscula zeta
%  \item[$ \in $] Pertence
\end{simbolos}
% ---

% ---
% inserir o sumario
% ---
\pdfbookmark[0]{\contentsname}{toc}
\tableofcontents*
\cleardoublepage
% ---



% ----------------------------------------------------------
% ELEMENTOS TEXTUAIS
% ----------------------------------------------------------
\textual

% ----------------------------------------------------------
% Introdução (exemplo de capítulo sem numeração, mas presente no Sumário)
% ----------------------------------------------------------
\chapter[Introdução]{Introdução}

	Os satélites artificiais amplamente utilizados e essenciais no dia-a-dia para diversas tarefas, como por exemplo para previsões meteorológicas, \textit{GPS} e televisões via satélite, são objetos que orbitam os planetas em trajetos circulares ou elípticos. Esses satélites, feitos pelo homem, são desenvolvidos especificamente para funções preestabelecidas que tornem possível alcançar objetivos maiores. 
	
	Esse formato de desenvolvimento individual faz o seu processo produtivo ser lento e com custos elevados, o que torna a alta tecnologia encontrada nos satélites restrita a pequenos grupos de engenheiros e cientistas. A combinação desses fatores acabou motivando, no final dos anos 90, os professores Jordi Puig-Suari e Bob Twiggs, a proporem o modelo do \textit{CubeSat}, que são satélites miniaturizados com tempo de desenvolvimento e custos bem abaixo dos satélites tradicionais.
	
	O presente trabalho apresenta o Sistema de Gerenciamento de Energia de um \textit{CubeSat}, ele é o subsistema responsável pela geração, transmissão e gerenciamento de energia, tendo por finalidade fornecer energia elétrica suficiente para o funcionamento dos demais subsistemas pertencentes a este satélite miniaturizado, como por exemplo o subsistema de comunicação, controle de atitude e computador de bordo.
	
	O processo de geração de energia depende da captação de luz solar suficiente para suprir a demanada energética do \textit{CubeSat}, além de ser capaz de realizar o carregamento de uma bateria. Essa bateria, que por sua vez, tem a capacidade de fornecer energia para todo o sistema, inclusive nos momentos nos quais o \textit{CubeSat} estiver na região de sombra da Terra.
	
	O Sistema de Gerenciamento de Energia foi totalmente dimensionado de forma a atender todos os pré-requisitos da construção de um \textit{CubeSat}, que futuramente poderá ser enviado para a realização de uma missão espacial.

% ----------------------------------------------------------
% Objetivo
% ----------------------------------------------------------
%\section{Objetivo}

	Esse projeto de conclusão de curso, faz parte do projeto de desenvolvimento de um \textit{CubeSat} do NSEE-IMT da Escola de Engenharia Mauá, que além de ajudar a fomentar a pesquisa e desenvolvimento de projetos para formar e capacitar alunos e pesquisadores na área aeroespacial, visa fornecer a energia necessária, com incidência direta ou não de luz solar, para garantir o sucesso de missões espaciais.
	
	O presente trabalho está dividido em \ref{ultimo_capitulo} capítulos contendo as seguintes abordagens.

	O capítulo 1 traz a apresentação do trabalho, suas principais características, inovações e os objetivos principais.

	O capítulo 2 apresenta um estudo histórico sobre os \textit{CubeSats}, como surgiram e o posicionamento do Brasil neste segmento. 

	O capítulo 3 faz um estudo sobre o estado da arte no segmento espacial, apresenta a proposta do estudo e faz a discretização dos componentes utilizados para o desenvolvimento do Sistema de Gerenciamento de Energia para \textit{CubeSat}.

	O capítulo 4 mostra os resultados e discussões dos dados obtidos no levantamento dos rendimento do subsistema proposto.

	O capítulo 5 trata das conclusões obtidas através das análises dos resultados dos ensaios realizados. 

% ----------------------------------------------------------
% REFERENCIAL TEÓRICO
% ----------------------------------------------------------
\chapter[Referencial Teórico]{Referencial Teórico}

	Os \textit{CubeSats}, atualmente, estão ganhando grande atenção acadêmica ao redor mundo, o que está contribuindo para a difusão do conhecimento e da tecnologia aeroespacial. A seguir será apresentado a definição de um \textit{CubeSat}, assim como as suas principais características de projeto, o seu surgimento e uma breve análise dos projetos brasileiros. 

\section[CubeSat]{\textit{CubeSat}}

	É um tipo de satélite miniaturizado usado em pesquisas espaciais. Por definição de projeto, elaborado pela \textit{California Polytechnic State University} (\textit{Cal Poly}), um \textit{CubeSat} deve possuir volume máximo de um litro, ou seja, ser um cubo de 10x10x10 cm e com massa máxima de até 1,3 kg.\textsuperscript{\cite{CubeSat}}
	
	Um \textit{CubeSat} com essas especificações são chamados de \textit{CubeSat} de 1U, ou seja, é um \textit{CubeSat} de 1 unidade, conforme a Figura \ref{Fig_Cubo}. Porém outras unidades podem ser adicionadas gerando os \textit{CubeSats} de 2U, 3U, 4U e etc.
	
	\begin{figure}[th]
		\caption{MODELO DE UM CUBESAT DE 1U}
		\label{Fig_Cubo}
		\centering
		\includegraphics[width=0.8\linewidth]{./figs/cubesat_01}
			
		\begin{small}
			FONTE: \textit{NASA}\textsuperscript{\cite{NASA}}
		\end{small}		
	\end{figure}
		
	O \textit{CubeSat} Mauá, proposto no NSEE-IMT, é um equipamento do modelo 3U, sendo as unidades distribuídas em unidade de comunicação, unidade de controle de atitude e unidade de potência, como apresentado na Figura \ref{Fig_Est_Cubo}.
	
	\begin{figure}[th]
		\caption{ESTRUTURA DO CUBESAT PROPOSTA PELO NSEE-IMT}
		\label{Fig_Est_Cubo}
		\centering
		\includegraphics[width=0.4\linewidth]{./figs/cubesat_02}
			
		\begin{small}
			FONTE: Especificação do produto CubeSat\textsuperscript{\cite{Corsi}}
		\end{small}		
	\end{figure}
	\pagebreak

\section[Surgimento]{Surgimento} 

	O primeiro projeto de um \textit{CubeSat} foi proposto em 1999 pelos professores Jordi Puig-Suari, da \textit{California Polytechnic State University}, e Bob Twiggs, da \textit{Stanford University}. O objetivo do projeto foi o de padronizar o \textit{design} de picosatélites, visando a redução de custos e de tempo de desenvolvimento, além de prover uma maior acessibilidade ao espaço e conseguir realizar lançamentos frequentes, o que é de inviável obtenção com os satélites de grande porte.\textsuperscript{\cite{CubeSat}}

\section[No Brasil]{No Brasil}

	Os projetos de picosatélites, nanosatélites e microsatélites se multiplicam a cada ano, não só no Brasil mas em todo o mundo. A \textit{Cal Poly} estima que atualmente o projeto \textit{CubeSat} conte com a colaboração internacional de mais de 100 universidades, colégios e de algumas empresas e governos.
	
	Atualmente vários projetos nessa área, de pequenos satélites para diversas áreas da pesquisa científica e tecnológica, estão em curso no Brasil e outros ainda em fase de discussão, dentre eles se destacam os projetos abaixo:
	
	\begin{itemize}
		\item \textbf{Tancredo 1}
		
			Picosatélite desenvolvido pelo grupo do professor Cândido Moura da Escola Tancredo Neves de Ubatuba, São Paulo. Primeiro satélite do Projeto UbatubaSat.\textsuperscript{\cite{UbatubaSat}}
			
		\item \textbf{AESP-14}		
		
			\textit{CubeSat} desenvolvido pelo grupo do Dr. Pedro Lacava, professor e coordenador do Curso de Engenharia Aeroespacial do Instituto Tecnológico de Aeronáutica (ITA).\textsuperscript{\cite{AESP14}}
			
		\item \textbf{NanoSatC-Br2}
		
			Nanosatélite em desenvolvimento pelo grupo coordenado pelo Dr. Nelson Schuch do Centro Regional Sul do INPE (CRS) e do Dr. Otávio Durão (INPE/SJC), em parceria com pesquisadores da Universidade Federal de Santa Maria (UFSM) do Rio Grande do Sul, em orbita desde 19/06/2014.\textsuperscript{\cite{INPE}}

		\item \textbf{14-BISat}

			Nanosatélite científico em desenvolvimento pelo grupo liderado pelo professor Cedric Salotto, coordenador do Centro de Referência em Sistemas Embarcados e Aeroespaciais (CRSEA) do Instituto Federal Fluminense (IFF) da cidade de Campos dos Goytacazes (RJ), em parceria com a empresa \textit{Tekever S/A} e a Faculdade de Engenharia da Universidade do Porto (FEUP), Portugal. Este projeto faz parte da missão internacional QB50.\textsuperscript{\cite{IFF}}
			
		\item \textbf{ITASAT-1}	

			Nanosatélite tecnológico em desenvolvimento pelo grupo liderado pelo Major Eloi Fonseca, professor do Curso de Engenharia Aeroespacial do Instituto Tecnológico de Aeronáutica (ITA) em parceria com a Agência Espacial Brasileira (AEB), Instituto Nacional de Pesquisas Espaciais (INPE-SJC, INPE-CRN e INPE-SM), Universidade do Vale do Rio dos Sinos (UNISINOS), Universidade Federal do Rio Grande do Norte (UFRN) e Universidade Federal de Santa Maria (UFSM).\textsuperscript{\cite{ITASAT}}	
			
	\end{itemize}

	Nesse capítulo foram apresentadas as premissas básicas de um projeto de \textit{CubeSat}, assim como o seu surgimento e  um resumo do segmento de nanossatélites no Brasil.


% ----------------------------------------------------------
% Materiais e Método
% ----------------------------------------------------------
\chapter[Materiais e Método]{Materiais e Método}

	Para auxiliar no projeto, as atividades foram divididas de forma a trazer, além do ganho teórico, uma maior dinâmica no desenvolvimento do mesmo. Essa etapa do projeto visou a máxima aquisição de dados possível sobre o tema proposto, foram abertas distintas frentes de trabalho para agilizar o entendimento teórico. Além disto, foram definidos os principais componentes que foram utilizados no protótipo, como por exemplo as baterias, fotocélulas, componentes passivos e semicondutores.
	
	Também foi possível identificar e conhecer, de forma mais profunda, possíveis testes que podem ser realizados no Sistema de Gerenciamento de Energia para \textit{CubeSat}, como os testes de radiação, temperatura e pressão. Esses testes são de extrema importância para a detecção de possíveis problemas que possam existir, pois uma vez que o \textit{CubeSat} for lançado nada mais poderá ser feito para reparar possíveis problemas de \textit{hardware}, sendo possível somente alguns ajustes de configuração de \textit{software}, como por exemplo alteração de parâmetros e de prioridades de tarefas.
	
% ----------------------------------------------------------
% Etapas
% ----------------------------------------------------------
\section[Topologia]{Topologia}

	A topologia do circuito foi o passo inicial para o desenvolvimento do Sistema de Gerenciamento de Energia para \textit{CubeSats}, onde foi definido através de um diagrama de blocos como funcionaria o circuito do projeto proposto.
	
	Diagrama de blocos é uma representação gráfica do modelo de um sistema e/ou processo complexo. É uma ferramenta extremamente difundida, utilizada como meio de referência para que pessoas não familiarizadas com o objeto de estudo, possam ter uma compreensão geral do projeto em questão.\textsuperscript{\cite{Diag_Blocos}}	
		
	Na Figura \ref{Fig_Diag_Blocos_Final}, pode ser visualizado o resultado final do diagrama de blocos utilizado no projeto.
	
	\begin{figure}[th]
		\caption{ÚLTIMA REVISÃO DO DIAGRAMA DE BLOCOS}
		\label{Fig_Diag_Blocos_Final}
		\centering
		\includegraphics[width=1.0\linewidth]{./figs/diag_blocos}
			
		\begin{small}
			FONTE: Elaborado pelos autores.
		\end{small}		
	\end{figure}
	\pagebreak
	
	A definição do diagrama de blocos do Sistema de Gerenciamento de Energia para \textit{CubeSats} foi o passo inicial para o desenvolvimento do projeto, pois todo o estudo posterior foi realizado baseado no funcionamento do circuito pensado.

	Para atingir o nível de detalhe exibido no diagrama de blocos da Figura \ref{Fig_Diag_Blocos_Final}, diversas horas de estudos foram dedicadas, sendo essas horas distribuídas entre os estudos que serão apresentados nas seções subsequentes, como também através de diversos \textit{brainstorms} realizados pelo grupo.
	
	Na Figura \ref{Fig_Diag_Blocos_Inicial}, pode ser visualizado a ideia embrionária do diagrama de blocos pensado para o desenvolvimento do projeto.
		
	\begin{figure}[th]
		\caption{VERSÃO INICIAL DO DIAGRAMA DE BLOCOS}
		\label{Fig_Diag_Blocos_Inicial}
		\centering
		\includegraphics[width=1.0\linewidth]{./figs/diag_blocos_inicial}
			
		\begin{small}
			FONTE: Elaborado pelos autores.
		\end{small}		
	\end{figure}	
	\pagebreak

	Inicialmente, foi previsto a utilização de 4 placas com fotocélulas, as quais iriam gerar uma tensão no carregador de baterias e no barramento interno. A partir desse barramento a energia chegaria aos conversores sendo convertidas para 3,3V, 5V, 12V e uma tensão não regulada (V\textsubscript{nreg}). Essas tensões provenientes dos conversores, passariam pelos sensores de tensão e corrente chegando ao barramento externo, alimentando todos os subsistemas do \textit{CubeSat}.
	
	Nessa topologia quando o \textit{CubeSat} estivesse em região de sombra solar, ou seja, quando não tivesse incidência direta de luz nas suas fotocélulas, o microcontrolador faria o chaveamento para a bateria de 4,2V, dessa forma a bateria assumiria a responsabilidade de gerar a energia necessária para o funcionamento do sistema até que o \textit{CubeSat} voltasse a receber incidência da luz solar.
	
	Alguns fatores foram determinantes para a alteração da topologia utilizada no projeto.
	
	\begin{minipage}{7cm}	
		\begin{itemize}
			\item	Condições do espaço;
			\item	Falta de redundância;
			\item	Eficiência dos conversores;
		\end{itemize}
	\end{minipage}
	\begin{minipage}{7cm}
		\begin{itemize}
			\item	Adição do conversor \textit{backup};
			\item	Troca da bateria de 4,2V para 7,4V.
		\end{itemize}
	\end{minipage}
	\pagebreak
	
	Já na versão final do diagrama, representado na Figura \ref{Fig_Diag_Blocos_Final}, o projeto foi subdividido em três partes, sendo elas, a placa de controle, placa do carregador e as quatro placas de fotocélulas (+X, -X, +Y e -Y). Nas seções seguintes serão exibidos os estudos que justificaram as alterações realizadas.
 
\section[Condições do espaço]{Condições do espaço}
	
	Para o desenvolvimento do \textit{CubeSat} é de extrema importância ter conhecimento das condições de operação que o equipamento irá operar.
	
	Os \textit{CubeSats}, operam em órbita terrestre baixa (\textit{LEO - Low Earth Orbit}). A órbita \textit{LEO} é a órbita que se encontra abaixo de 2.000 km do nível do mar, os objetos que situam-se nela, geralmente, ficam entre 320 até 800 km da superfície terrestre, muito diferente dos satélites tradicionais que operam em órbita geoestacionária, cuja a distância é de 35.796 km em relação ao nível do mar.\textsuperscript{\cite{LEO}}\textsuperscript{\cite{GEO}}
	
	\begin{figure}[th]
		\caption{PRINCIPAIS TIPOS DE ÓRBITAS}
		\label{Fig_Orbitas}
		\centering
		\includegraphics[width=0.65\linewidth]{./figs/cubesat_03}
			
		\begin{small}
			FONTE: Civil Air Patrol\textsuperscript{\cite{CAP}}
		\end{small}		
	\end{figure}
	
	Satélites situados na órbita \textit{LEO} viajam em velocidades de aproximadamente 27.400 km/h ou 8 km/s, o que representa uma volta ao longo da Terra a cada 90 minutos. Já os satélites geoestacionários precisam ter uma velocidade que façam que eles acompanham sempre o mesmo ponto da Terra, por isso as velocidades deles são de aproximadamente 11.068 km/h ou 3 km/s. O planeta Terra tem uma velocidade de rotação de aproximadamente 1.669,8 km/h ou 0,5 km/s.\textsuperscript{\cite{LEO}}\textsuperscript{\cite{GEO}}
	
	Outras características de destaque para a órbita \textit{LEO} são as condições de temperatura, variando de -170 ºC a 123 ºC e de pressão variando de 10\textsuperscript{-4} Pa a 10\textsuperscript{-6} Pa.\textsuperscript{\cite{LEO}}
	
	Órbitas inferiores a esta não apresentam muita estabilidade e são alvos de arrastamento atmosférico, que é a força de fricção que atua sobre o foguete ou satélite, cuja principal causa é a fricção entre as moléculas do ar e a superfície do foguete ou satélite.\textsuperscript{\cite{NASA2}}
	
	A partir desse estudo preliminar foram definidos os critérios para a seleção dos componentes do projeto.	
	
\section[Definição dos componentes]{Definição dos componentes}

	A seguir será explicado de forma mais detalhada como foram realizadas as escolhas dos principais componentes do projeto. Importante ressaltar que os projetos de \textit{CubeSats} possuem como premissa o conceito de ser um projeto de baixo custo, porém o referencial do custo utilizado são os custos de projetos de grandes satélites.
	
	Para a utilização dos componentes que suportem as condições impostas  no meio espacial, alguns fabricantes possuem linhas de produtos voltadas para utilização de componentes aeroespaciais que possuem um custo mais elevado em comparação aos componentes utilizados no mercado comum.

\subsection[Bateria]{Bateria} \label{Sec_Bateria}

	
	As baterias têm a função principal de armazenar carga e fornecer energia para todos os subsistemas do \textit{CubeSat}, incluindo nos momentos no qual o equipamento estiver situado em regiões de sombras solares, por exemplo atrás do planeta Terra. Nessas regiões, não há incidência de luz solar portanto as fotocélulas não irão gerar energia no carregador da bateria. Dessa forma a bateria estará somente consumindo a energia armazenada até que o \textit{CubeSat} volte a ter incidência de luz solar para recomeçar o ciclo de carregamento, caso contrário o sistema será desligado e o equipamento virará apenas lixo espacial.	
	
	Algumas das premissas básicas de projeto para a definição da bateria estão relacionadas com o seu poder de armazenamento de carga e o seu dimensional reduzido. Essas baterias ficam alocadas no interior do \textit{CubeSat}, por isso a importância do dimensional reduzido, além disso não podem ser baterias com peso elevado, uma vez que a definição do projeto diz que os \textit{CubeSats} não podem ultrapassar 1,3 kg.
	
	Para a definição da bateria foi realizado um comparativo (Tabela \ref{Tab_Bateria}) dos tipos mais comuns de baterias encontradas no mercado, sendo elas de: níquel cádmio (NiCd), níquel-hidreto metálico (NiMh), íon-lítio (Li-Íon), íon-lítio polímero (Li-Íon polímero) e chumbo (Pb).
		
	\begin{table}[th]
		\caption{COMPARATIVO DOS PRINCIPAIS TIPOS DE BATERIAS}
		\label{Tab_Bateria}
		\begin{tabular}{p{3.5cm}|p{1.6cm}|p{1.6cm}|p{1.6cm}|p{2cm}|p{1.6cm}}
	 	& \textbf{NiCd} & \textbf{NiMh} & \textbf{Li-Íon} & \textbf{Li-Íon polímero} & \textbf{Pb} \\
	 	\hline
		\textbf{Densidade energética (Wh/kg)} & 45-80 & 60-120 & 110-160 & 100-130 & 30-50 \\
	 	\hline
	 	\textbf{Resistência interna (m$\Omega$)} & 100-200\textsuperscript{a} & 200-300\textsuperscript{a} & 150-250\textsuperscript{b} & 200-300\textsuperscript{b} & <100\textsuperscript{c} \\
	 	\hline
	 	\textbf{Ciclo de vida} & 1500 & 500-1000 & 500-1000 & 300-500 & 200-300 \\
		\hline
		\textbf{Tempo para carga rápida} & 1 hr & 2-4 hr & 2-4 hr & 2-4 hr & 8-16 hr \\
		\hline
		\textbf{Tolerância para sobrecarga} & Moderada & Baixa & Muito baixa & Baixa & Alta \\
		\hline
		\textbf{Auto-descarga mensal} & 20\% & 30\% & 10\% & 10\% & 5\% \\
		\hline
		\textbf{Tensão da célula (V)} & 1,25 & 1,25 & 3,6 & 3,6 & 2 \\
		\hline
		\textbf{Corrente de carga (In)} & 1,0 & 0,5 & 1,0 & 1,0 & 0,2 \\
		\hline
		\textbf{Temperatura de operação (ºC)} & -40 a 60 & -20 a 60 & -20 a 60 & 0 a 60 & -20 a 60 \\
		\hline
		\textbf{Manutenção (dias)} & 30-60 & 60-90 & --- & --- & 90-180 \\
		\hline
		\textbf{Comparação de custo pack 7,2 V (dólar)} & 50,00 & 60,00 & 100,00 & 100,00 & 25,00 \\
		\hline
		\textbf{Custo por ciclo (dólar)} & 0,04 & 0,12 & 0,14 & 0,29 & 0,10 \\
		\hline 
		\end{tabular}
	\centering
	\begin{small}
	\vspace{3pt}
		FONTE: STA Eletrônica\textsuperscript{\cite{sta}}
	\end{small}
	
	\begin{footnotesize}
		NOTA: a: pack 6V; b: pack 7,2V; c: pack 12V.
	\end{footnotesize}

\end{table}
	\pagebreak
	Dentre os modelos comparados na Tabela 1, foi escolhida a bateria do tipo íon-lítio. Essa definição foi baseada pelo alto ciclo de vida, alta densidade energética, baixa auto-descarga mensal, a falta de necessidade de manutenção. Alguns dos pontos negativos desse tipo de bateria é baixa tolerância para sobrecargas, tempo para carga rápida e preço elevado em comparação aos demais tipos.
	
	Foram utilizadas duas baterias de duas células que possuem 7,4 V e 2000 mAh, sendo uma para o conjunto principal e a outra para o conjunto de redundância do Sistema de Gerenciamento de Energia para \textit{CubeSat}. Na Figura \ref{Fig_Bat_Sel}, é possível visualizar a bateria selecionada.
	
	\begin{figure}[th]
		\caption{BATERIA DE ÍON-LÍTIO SELECIONADA}
		\label{Fig_Bat_Sel}
		\centering
		\includegraphics[width=0.8\linewidth]{./figs/cubesat_04}
			
		\begin{small}
			FONTE: Fotos tiradas pelos autores.
		\end{small}
	\end{figure}
	\pagebreak
\subsubsection[Princípio de funcionamento]{Princípio de funcionamento}
	
	As baterias de íon-lítio, têm esse nome devido ao seu princípio de funcionamento o qual consiste no movimento dos íons de lítio por meio de um solvente não aquoso.\textsuperscript{\cite{BraEsc}}	
	
	Na bateria de íon-lítio, o ânodo é feito de carbono enquanto o cátodo de algum óxido de metal que contenha lítio. Ambos os terminais ficam imersos a um eletrólito e separados por um separador. Quando adicionado uma carga à esses terminais os íons de lítio migram do ânodo para o cátodo, se os terminais são conectados a um carregador o processo ocorre de forma inversa,  ou seja, os íons de lítio partem do cátodo para o ânodo.\textsuperscript{\cite{sony}}
	
	Para haver essa movimentação dos íons de lítio, o que deixa a bateria com a característica de ser recarregável, nenhum dos terminais (ânodo ou cátodo) dessa bateria pode ser puramente do metal lítio.\textsuperscript{\cite{sony}}
	
	Na Figura \ref{Fig_PF_Bat}, pode ser visto o mecanismo de carga e descarga de uma bateria de íon-lítio.
	
	\begin{figure}[th]
		\caption{PRINCÍPIO DE FUNCIONAMENTO DA BATERIA DE ÍON-LÍTIO}
		\label{Fig_PF_Bat}
		\centering
		\includegraphics[width=0.8\linewidth]{./figs/funcionamento_bateria}
			
		\begin{small}
			FONTE: Sony Corporation\textsuperscript{\cite{sony}}
		\end{small}		
	\end{figure}
	\pagebreak
	O lítio (Li) é considerado o metal mais leve existente na Terra (desconsiderando os feitos em laboratório), por essa razão que as baterias de íon-lítio possuem baixo peso, o que é de fundamental importância para o projeto do \textit{CubeSat}. Por se tratar de um dos tipos de bateria mais comuns, sendo amplamente encontrado em \textit{smartphones} e \textit{notebooks}, acaba tendo um impacto positivo nos custos de aquisição das mesmas.\textsuperscript{\cite{TecMundo}}
	
\subsection[Fotocélulas]{Fotocélulas}

	As fotocélulas são fundamental importância para o sistema, uma vez que elas são as responsáveis pela captação da luz solar que irá gerar a energia necessária para o funcionamento do \textit{CubeSat}. É importante que elas possuam rendimento elevado, pois devido as condições impostas pela órbita \textit{LEO}, na qual o \textit{CubeSat} irá ficar um terço do período de translação em regiões de sombra, ou seja, estará sem a incidência direta de luz solar.
	
	Inicialmente as fotocélulas eram fabricadas a partir do Silício, na década de 1960 essas células apresentavam aproximadamente 10\% de eficiência, atualmente essa eficiência chega próximo aos 20\%. No final da década de 1980 começaram a ser produzidas fotocélulas de Arseneto de Gálio, as quais atingiram uma eficiência superior a 20\%, essa nova composição começou porque as fotocélulas de Silício chegaram a ser produzidas com Silício de alta pureza (99,999999\%). Após as primeiras células de Arseneto de Gálio foi adicionado a sua composição o Índio, obtendo ótimos resultados.\textsuperscript{\cite{Fatemi}}
	
	Nas fotocélulas de uma única junção, esse excesso de energia se converte em calor, ou seja, é uma perda de energia absorvida. A técnica de multijunções atenua o problema da perda de energia quando um fóton de energia maior do que a energia do semicondutor é absorvido. Nessas fotocélulas cada junção funciona como um filtro, ou seja, cada junção absorve uma parte do espectro de luz e deixa o restante passar para a próxima junção, dessa maneira o rendimento é muito melhor devido ao alto aproveitamento do espectro.\textsuperscript{\cite{Fatemi}}
	
\subsubsection[Comparativo das tecnologias]{Comparativo das tecnologias}

	Devido as opções disponíveis de tecnologia foram realizados alguns estudos com o intuito de se obter qual seria o melhor tipo de fotocélula para o seguimento do projeto. A seguir serão comparados as principais métricas necessárias para uma fotocélula de aplicação espacial.
	
	\begin{figure}[th]
		\caption{COMPARATIVO DE EFICIÊNCIA DAS FOTOCÉLULAS}
		\label{Fig_cell_comp1}
		\centering
		\includegraphics[width=0.6\linewidth]{./figs/cell_comp1}
			
		\begin{small}
			FONTE: Solar array trades between very high-efficiency multi-junction ans Si space solar cells.\textsuperscript{\cite{Fatemi}}
		\end{small}		
	\end{figure}
	
	Na Figura \ref{Fig_cell_comp1} é visível que as fotocélulas de tripla junção são mais eficientes e resistentes a radiação do que as demais tecnologias analisadas (Sílicio e dupla junção).
	
	\begin{figure}[th]
		\caption{COMPARATIVO DE POTÊNCIA DAS FOTOCÉLULAS}
		\label{Fig_cell_comp2}
		\centering
		\includegraphics[width=0.6\linewidth]{./figs/cell_comp2}
			
		\begin{small}
			FONTE: Solar array trades between very high-efficiency multi-junction ans Si space solar cells.\textsuperscript{\cite{Fatemi}}
		\end{small}		
	\end{figure}	
	\pagebreak
	Já na Figura \ref{Fig_cell_comp2}, é realizado um comparativo da potência pela área e da potência pela massa. Nesse comparativo pode-se ver, novamente, que as fotocélulas de tripla função levam vantagem em relação as demais quando analisado a Potência (W) pela área (m\textsuperscript{2}), pois essa tecnologia permite um melhor aproveitamento do espectro de luz absorvido, conforme pode ser visto na Figura \ref{Fig_cell_mult}. 
	
	\begin{figure}[th]
		\caption{ESQUEMA DE UMA CÉLULA MULTIJUNÇÃO E SEU ESPECTRO}
		\label{Fig_cell_mult}
		\centering
		\includegraphics[width=1.05\linewidth]{./figs/cell_mult}
			
		\begin{small}
			FONTE: Solar array trades between very high-efficiency multi-junction ans Si space solar cells.\textsuperscript{\cite{Fatemi}}
		\end{small}		
	\end{figure}	
	
	E por fim, quando o comparativo é em relação a massa das fotocélulas, a de Sílicio leva vantagem devido ao menor peso que possui comparada as demais, como visto na Figura \ref{Fig_cell_comp3}
	
	\begin{figure}[th]
		\caption{COMPARATIVO DE DIMENSÃO DAS FOTOCÉLULAS}
		\label{Fig_cell_comp3}
		\centering
		\includegraphics[width=0.6\linewidth]{./figs/cell_comp3}
			
		\begin{small}
			FONTE: Solar array trades between very high-efficiency multi-junction ans Si space solar cells.\textsuperscript{\cite{Fatemi}}
		\end{small}		
	\end{figure}
	
	Com base nos comparativos apresentados nas Figuras \ref{Fig_cell_comp1}, \ref{Fig_cell_comp2} e \ref{Fig_cell_comp3}, foi definido que as fotocélulas de tripla junção seriam as mais adequadas para a continuidade do projeto.
	
\subsubsection[Princípio de funcionamento]{Princípio de funcionamento}

	As fotocélulas são um exemplo de aplicação prática do efeito fotoelétrico, descoberto por Heinrich Rudolf Hertz, em 1887 e explicado por Albert Einstein, em 1905.\textsuperscript{\cite{celula}}
	
	Quando uma grande quantidade de fótons é incidida em uma fotocélula a energia é absorvida. Essa absorção de energia permite que os átomos dos elementos que constituem a célula liberem elétrons, o espaço liberado é preenchido por outro elétron de uma camada inferior do semicondutor. Essa movimentação de elétrons, faz com que um dos lados da célula tenha uma concentração maior de elétrons, o que origina a diferença de potencial entre os lados, conforme pode ser visto na Figura \ref{Fig_PF_Cell}.\textsuperscript{\cite{celula2}}
	
	\begin{figure}[th]
		\caption{PRINCÍPIO DE FUNCIONAMENTO DE UMA FOTOCÉLULA}
		\label{Fig_PF_Cell}
		\centering
		\includegraphics[width=0.6\linewidth]{./figs/fotocelula}
			
		\begin{small}
			FONTE: Sapa Solar\textsuperscript{\cite{celula2}}
		\end{small}		
	\end{figure}

\subsubsection[VIS Technology]{\textit{VIS Technology}}
	
	Inicialmente foi indicado pelo Engenheiro Rafael Corsi, do NSEE-IMT, o contato da empresa \textit{Vis Technology}, um empresa nacional, localizada em São Paulo, que desenvolve projetos com energias renováveis. Porém as fotocélulas utilizadas por eles são para aplicações industriais. Essas fotocélulas possuem um rendimento entre 10\% e 12\%, muito abaixo comparado ao rendimento com as próprias de aplicações aeroespaciais, além de terem um dimensional maior, conforme pode ser visualizado na Figura \ref{Fig_Cell_Vis}.
	
	\begin{figure}[th]
		\caption{FOTOCÉLULAS DA \textit{VIS TECHNOLOGY}}
		\label{Fig_Cell_Vis}
		\centering
		\includegraphics[width=0.6\linewidth]{./figs/cell_vis}
			
		\begin{small}
			FONTE: Foto tirada pelos autores.
		\end{small}		
	\end{figure}
	
	Mesmo sabendo dessas limitações, foram obtidas algumas amostras. Essas amostras serviram para um primeiro contato com essa tecnologia, conseguindo realizar alguns ensaios a fim de se obter uma familiaridade maior com o componente, conforme pode ser visto na Figura \ref{Fig_Cell_Setup}.
	
	\begin{figure}[th]
		\caption{\textit{SET UP} PARA TESTE DAS FOTOCÉLULAS DA \textit{VIS TECHNOLOGY}}
		\label{Fig_Cell_Setup}
		\centering
		\includegraphics[width=0.7\linewidth]{./figs/setup}
			
		\begin{small}
			FONTE: Foto tirada pelos autores.
		\end{small}		
	\end{figure}
	\pagebreak
	Após uma breve familiarização com as fotocélulas, foram analisados diversos projetos de \textit{CubeSats} e identificados os principais fornecedores de fotocélulas (\textit{SpectroLab}, \textit{Emcore} e \textit{AzurSpace Solar}) para aplicações aeroespaciais.
	
	Foi realizado um estudo apurado dos principais tipos de fotocélulas disponíveis nos portfólios desses fornecedores, visando identificar os modelos que se melhor ajustavam no desenvolvimento do \textit{CubeSat}.
	
\subsubsection[SpectroLab]{\textit{SpectroLab}}
	
	A \textit{SpectroLab}, empresa subsidiária da \textit{The Boeing Company}, é a fabricante líder mundial de células solares de multi-junção de alta eficiência e de painéis solares. A empresa é sediada nos Estados Unidos, mais especificamente em Los Angeles, Califórnia.\textsuperscript{\cite{SpectroLab}}
	
	\begin{table}[th]
	\caption{COMPARATIVO DAS FOTOCÉLULAS DA \textit{SPECTROLAB}}
	\label{Tab_Spectro_Comp}
	\begin{tabular}{p{2.5cm}|p{3.1cm}|p{3.1cm}|p{3.1cm}|p{3.1cm}}
		\textbf{Modelo} & \textit{\textbf{PV UTJ Cell}} & \textit{\textbf{PV XTJ Cell}} & \textit{\textbf{PV NM TASC ITJ}} & \textit{\textbf{PV ITJ Cell}} \\
		\hline
		\textbf{Rendimento} & 28,3\% & 29,5\% & 24\% a 30\% & 26,8\% \\
		\hline
		\textbf{Material} & GaInP2/GaAs/Ge & GaInP2/GaAs/Ge & GaInP2/GaAs/Ge & GaInP2/GaAs/Ge\\
		\hline
		\textbf{Tensão} & 2,660 V & 2,633 V & 2,520 V & 2,565 V\\
		\hline
		\textbf{Corrente} & 454 mA & 472 mA & 31 mA & 441 mA\\
		\hline
		\textbf{Dimensional} & 26,62 cm\textsuperscript{2} & 26,62 cm\textsuperscript{2} & 2,277 cm\textsuperscript{2} & 31 cm\textsuperscript{2}\\
		\hline
		\textbf{Peso} & 84 mg/cm\textsuperscript{2} & 84 mg/cm\textsuperscript{2} & 0,234 g & 84 mg/cm\textsuperscript{2}\\
	\end{tabular}
	
	\begin{small}
	\vspace{3pt}
		FONTE: Elaborada pelos autores através de informações coletadas nos \textit{datasheets} dos produtos.
	\end{small}
	\begin{footnotesize}
		NOTA: Condições de teste: \textit{AM} 0, \textit{WRC} = 135,3 mW/cm\textsuperscript{2}, T = 28ºC.
	\end{footnotesize}
	\end{table}

	Avaliando os dados da Tabela \ref{Tab_Spectro_Comp}, os modelos \textit{PV UTJ Cell} e \textit{PV XTJ Cell}, foram os mais indicados para a aplicação, devido ao alto rendimento (superior a 28\%) apresentado
	
	\begin{figure}[th]
		\caption{MODELO DE FOTOCÉLULA DA \textit{SPECTROLAB}}
		\label{Fig_Cell_Spectro}
		\centering
		\includegraphics[width=1.0\linewidth]{./figs/UTJ}
			
		\begin{small}
			FONTE: \textit{SpectroLab}\textsuperscript{\cite{SpectroLab2}}
		\end{small}		
	\end{figure}

\subsubsection[Emcore]{\textit{Emcore}}

	Em dezembro de 2014, a \textit{Emcore} foi comprada pela \textit{SolAero Technologies}, que é uma das fabricantes líderes mundial de alta eficiência, células solares e painéis solares para aplicações espaciais. Assim como a \textit{SpectroLab}, está sediada nos Estados Unidos, porém no município de Albuquerque, Novo México.\textsuperscript{\cite{Emcore}}\textsuperscript{\cite{Emcore2}}
	
	\begin{table}[th]
	\caption{COMPARATIVO DAS FOTOCÉLULAS DA \textit{EMCORE}}
	\label{Tab_Emcore_Comp}
	\begin{tabular}{p{2.5cm}|p{3.1cm}|p{3.1cm}|p{3.1cm}|p{3.1cm}}
		\textbf{Modelo} & \textit{\textbf{ATJ PV Cell}} & \textit{\textbf{BTJ PV Cell}} & \textit{\textbf{BTJM PV Cell}} & \textit{\textbf{ZTJ PV Cell}} \\
		\hline
		\textbf{Rendimento} & 27,5\% & 28,5\% & 28\% & 29,5\% \\
		\hline
		\textbf{Material} & GaInP/GaAs/Ge & GaInP/GaAs/Ge & GaInP/GaAs/Ge & GaInP/GaAs/Ge\\
		\hline
		\textbf{Tensão} & 2,60 V & 2,70 V & 2,69 V & 2,73 V\\
		\hline
		\textbf{Corrente} & 454 mA & 455 mA & 454 mA & 467 mA\\
		\hline
		\textbf{Dimensional} & 26,6 cm\textsuperscript{2} & 26,6 cm\textsuperscript{2} & 26,6 cm\textsuperscript{2} & 26,6 cm\textsuperscript{2}\\
		\hline
		\textbf{Peso} & 84 mg/cm\textsuperscript{2} & 84 mg/cm\textsuperscript{2} & 84 mg/cm\textsuperscript{2} & 84 mg/cm\textsuperscript{2}\\
	\end{tabular}
	
	\begin{small}
	\vspace{3pt}
		FONTE: Elaborada pelos autores através de informações coletadas nos \textit{datasheets} dos produtos.
	\end{small}
	\begin{footnotesize}
		NOTA: Condições de teste: \textit{AM} 0, \textit{WRC} = 135,3 mW/cm\textsuperscript{2}, T = 28ºC.
	\end{footnotesize}
	\end{table}
	\pagebreak
	Avaliando os dados da Tabela \ref{Tab_Emcore_Comp}, os modelos \textit{BTJ PV Cell}, \textit{BTJM PV Cell} e \textit{ZTJ PV Cell}, foram os mais indicados para a aplicação, devido ao alto rendimento (superior a 28\%) apresentado.
	
	\begin{figure}[th]
		\caption{MODELO DE FOTOCÉLULA DA \textit{EMCORE}}
		\label{Fig_Cell_Emcore}
		\centering
		\includegraphics[width=0.5\linewidth]{./figs/ZTJ}
			
		\begin{small}
			FONTE: \textit{SolAero Technologies}\textsuperscript{\cite{Emcore3}}
		\end{small}		
	\end{figure}
	
\subsubsection[AzurSpace]{\textit{AzurSpace}}

	A \textit{AzurSpace} é a líder européia no desenvolvimento e produção de células solares multi-junção para aplicações espaciais e terrestres, com quase 50 anos de experiência no mercado. A empresa localiza-se em Heilbronn, cidade em Baden-Württemberg, Alemanha.\textsuperscript{\cite{AzurSpace}}
	
	\begin{table}[th]
	\caption{COMPARATIVO DAS FOTOCÉLULAS DA \textit{AZURSPACE}}
	\label{Tab_Azur_Comp}
	\begin{tabular}{p{2.5cm}|p{3.1cm}|p{3.1cm}|p{3.1cm}|p{3.1cm}}
		\textbf{Modelo} & \textit{\textbf{3G30C}} & \textit{\textbf{3G30C-Large}} & \textit{\textbf{3G30C-Large-120x60}} & \textit{\textbf{3G28C}} \\
		\hline
		\textbf{Rendimento} & 30\% & 30\% & 30\% & 28\% \\
		\hline
		\textbf{Material} & GaInP/GaAs/Ge & GaInP/GaAs/Ge & GaInP/GaAs/Ge & GaInP/GaAs/Ge\\
		\hline
		\textbf{Tensão} & 2,700 V & 2,700 V & 2,700 V & 2,667 V\\
		\hline
		\textbf{Corrente} & 520,2 mA & 1041 mA & 1186 mA & 506 mA\\
		\hline
		\textbf{Dimensional} & 30,18 cm\textsuperscript{2} & 60,36 cm\textsuperscript{2} & 68,76 cm\textsuperscript{2} & 30,18 cm\textsuperscript{2}\\
		\hline
		\textbf{Peso} & 86 mg/cm\textsuperscript{2} & 114 mg/cm\textsuperscript{2} & 130 mg/cm\textsuperscript{2} & 86 mg/cm\textsuperscript{2}\\
	\end{tabular}
	
	\begin{small}
	\vspace{3pt}
		FONTE: Elaborada pelos autores através de informações coletadas nos \textit{datasheets} dos produtos.
	\end{small}
	\begin{footnotesize}
		NOTA: Condições de teste: \textit{AM} 0, \textit{WRC} = 1367 W/m\textsuperscript{2}, T = 28 ºC.
	\end{footnotesize}
	\end{table}	
	
	Avaliando os dados da Tabela \ref{Tab_Azur_Comp}, os modelos \textit{30G28C} e \textit{30G30C}, foram os mais indicados para a aplicação, devido ao alto rendimento (superior a 28\%) apresentado. Os demais modelos não atenderam o dimensionamento adequado para o projeto.
	
	\begin{figure}[th]
		\caption{MODELO DE FOTOCÉLULA DA \textit{AZURSPACE}}
		\label{Fig_Cell_Azur}
		\centering
		\includegraphics[width=0.5\linewidth]{./figs/ZTJ}
			
		\begin{small}
			FONTE: \textit{AzurSpace}\textsuperscript{\cite{AzurSpace2}}
		\end{small}		
	\end{figure}	
	
\subsubsection[Análise de preços]{Análise de preços}

	Selecionados os modelos dos principais fornecedor, foi solicitado um orçamento das fotocélulas para realizar o comparativo dos preços e, por fim, realizar os procedimentos necessários para a aquisição das mesmas.
	
	O contato com a \textit{SpectroLab} não teve obteve sucesso, uma vez que a empresa não respondeu nenhuma das tentativas de contato realizadas. Já a solicitação do orçamento da \textit{SolAero Technologies} (antiga \textit{Emcore}) não caminhou conforme esperado, uma vez que uma das políticas de vendas da empresa é de um pedido mínimo de \$7.500,00.
	
	A \textit{AzurSpace} enviou um orçamento dos modelos solicitados, conforme Tabela \ref{Tab_Orc_Cell}.
	
	\begin{table}[th]
	\caption{ORÇAMENTO DAS FOTOCÉLULAS DA \textit{AZURSPACE}}
	\label{Tab_Orc_Cell}
	\centering
	\begin{tabular}{p{3.0cm}|p{3.0cm}|p{3.0cm}|p{3.0cm}}
		\textbf{Modelo} & \textbf{Valor unitário} & \textbf{Frete} & \textit{\textbf{Lead time}}\\
		\hline
		3G28C & \euro 193,00 & \euro 195,00 & 8 a 10 semanas\\
		3G30C & \euro 198,00 & \euro 195,00 & 8 a 10 semanas\\

	\end{tabular}
	
	\begin{small}
	\vspace{3pt}
		FONTE: Elaborada pelos autores através do orçamento recebido pela \textit{AzurSpace}.
	\end{small}
	\end{table}	
	
\subsubsection[TrisolX]{TrisolX}	
	
	Devido ao alto custo e o grande \textit{lead time} da \textit{AzurSpace}, novas pesquisas foram feitas na tentativa de encontrar um novo fornecedor de fotocélulas com um custo mais acessível. Após uma busca detalhada em diversos grupos de discussões sobre o desenvolvimento de \textit{CubeSats}, foi encontrada no \textit{LinkedIn} a empresa \textit{TrisolX}.
	
	A \textit{TrisolX} é uma pequena empresa, localizada em Nova Iorque, que oferece uma alternativa acessível para projetos com orçamentos reduzidos. A empresa, vende a \textit{TrisolX Solar Wings}, que são fotocélulas cortadas a partir do modelo \textit{3G28C} da \textit{AzurSpace}.
	
	É o mesmo produto da \textit{AzurSpace}, porém com tamanho e formato diferentes, conforme Figura \ref{Fig_Cell_TrisolX}.
	
	\begin{figure}[th]
		\caption{MODELO DE FOTOCÉLULA DA \textit{TRISOLX}}
		\label{Fig_Cell_TrisolX}
		\centering
		\includegraphics[width=0.5\linewidth]{./figs/TrisolX}
			
		\begin{small}
			FONTE: \textit{TrisolX}\textsuperscript{\cite{TrisolX}}
		\end{small}		
	\end{figure}	

	Foi solicitado um orçamento à \textit{TrisolX}, que prontamente foi recebido conforme Tabela \ref{Tab_Orc_TrisolX}.
	
	\begin{table}[th]
	\caption{ORÇAMENTO DAS FOTOCÉLULAS DA \textit{TRISOLX}}
	\label{Tab_Orc_TrisolX}
	\centering
	\begin{tabular}{p{3.5cm}|p{2.5cm}|p{1.5cm}|p{1.5cm}|p{2.0cm}}
		\textbf{Pacote} & \textbf{Quantidade} & \textbf{Valor} & \textbf{Frete} & \textbf{Prazo de entrega}\\
		\hline
		\textit{Sample Pack} & 5 células & \$25,00 &  \$68,00 & 2 semanas\\
		\textit{Starter Pack} & 25 células & \$100,00 &  \$68,00 & 2 semanas\\
		\textit{Development Pack} & 100 células & \$400,00 & \$68,00 & 2 semanas\\
	\end{tabular}
	
	\begin{small}
	\vspace{3pt}
		FONTE: Elaborada pelos autores através do orçamento recebedido pela \textit{TrisolX}.
	\end{small}
	
	\begin{footnotesize}
		NOTA: O modelo das fotocélulas comercializado pela \textit{TrisolX} é o \textit{3G28C} da \textit{AzurSpace}.
	\end{footnotesize}
	\end{table}	
	
	Tendo como objetivo inicial fazer a validação do sistema como um todo, realizar testes de pressão, temperatura e radiação, para fazer um levantamento completo do funcionamento do sistema, foi solicitado para a \textit{TrisolX} o \textit{Starter Pack}. Dessa forma foi possível ganhar experiência no manuseio das fotocélulas e validar o sistema, para depois fazer a aquisição das fotocélulas mais robustas da \textit{AzurSpace}.
	
\subsection[Componentes passivos]{Componentes passivos}

	Os componentes passivos, são os componentes eletrônicos que não aumentam a intensidade da tensão ou da corrente de um circuito eletrônico, ou seja, são os resistores, indutores, capacitores e memristores. No desenvolvimento do projeto não foram utilizados os memristores.\textsuperscript{\cite{Passivo}}
	
	Resistores são componentes utilizados para controlar a intensidade da corrente elétrica que passa no circuito. Capacitores são componentes que armazenam e liberam cargas elétricas por meio da tensão elétrica. Indutor são componentes que utilizam o magnetismo para armazenar e liberar cargas por meio da corrente elétrica.

	É de extrema importância que os componentes passivos atendam algumas premissas para a utilização no projeto. Esses componentes precisam possuir uma grande faixa de temperatura de operação, uma certa tolerância a radiação e um dimensional pequeno (\textit{SMD}).
	
	Os capacitores, devido as baixas pressões encontradas no espaço, não podem ser do modelo eletrolítico, ou seja, precisam ser ou cerâmicos ou de tântalo. Os capacitores cerâmicos geralmente são de 0,5 pF até 470 nF com tensão de isolação de 25 V ou 50 V. Para esses capacitores tomou-se o cuidado de selecionar os capacitores com baixo ESR e com o coeficiente de temperatura X7R, devido a sua grande faixa de temperatura de operação, conforme indicado na Figura \ref{Fig_Cap}.\textsuperscript{\cite{x7r}}
	
	\begin{figure}[th]
		\caption{COEFICIENTES DE TEMPERATURA DOS CAPACITORES CERÂMICOS}
		\label{Fig_Cap}
		\centering
		\includegraphics[width=1.0\linewidth]{./figs/x7r}
			
		\begin{small}
			FONTE: PY2BBS\textsuperscript{\cite{x7r}}
		\end{small}		
	\end{figure}	
	
	Além dessa característica, os capacitores X7R apresentam um bom comportamento com a variação de temperatura, comparado com os capacitores Y5V (outro modelo amplamente encontrado no mercado), conforme pode ser visto na Figura \ref{Fig_Cap_Temp}.
	
	\begin{figure}[th]
		\caption{VARIAÇÃO DA CAPACITÂNCIA EM FUNÇÃO DA TEMPERATURA}
		\label{Fig_Cap_Temp}
		\centering
		\includegraphics[width=1.0\linewidth]{./figs/y5v}
			
		\begin{small}
			FONTE: Johanson Dielectrics\textsuperscript{\cite{y5v}}
		\end{small}		
	\end{figure}
	\pagebreak

	Já os capacitores de tântalo apresentam valores de 0,22 pF até 100 $\mu$F, esse tipo de capacitor possui baixa corrente de fuga e baixas perdas, além de uma vida útil maior comparado aos eletrolíticos.\textsuperscript{\cite{x7r}}

\subsection[Semicondutores]{Semicondutores}

	Elementos semicondutores são os elementos químicos que possuem na sua camada de valência quatro elétrons. A condutividade desses elementos depende da temperatura na qual estão submetidos. Os principais elementos utilizados na construção de semicondutores, por exemplo diodos, transistores e circuitos integrados, são o silício e o germânio. A grande maioria dos semicondutores são construídos com silício, uma vez que o germânio é extremamente sensível a variações de temperatura.\textsuperscript{\cite{semicondutores}}
	
	Os semicondutores formam bandas de energia, como a banda de condução (BC) e a banda de valência (BV), essas bandas são separadas por lacunas. A banda de condução vai estar vazia e a banda de valência totalmente preenchida quando estiver com uma temperatura de 0K, como pode ser visto na Figura \ref{Fig_bandas}.\textsuperscript{\cite{semicondutores2}}
	
	\begin{figure}[th]
		\caption{BANDAS DE ENERGIA DE UM SEMICONDUTOR}
		\label{Fig_bandas}
		\centering
		\includegraphics[width=0.3\linewidth]{./figs/banda}
			
		\begin{small}
			FONTE: InfoEscola\textsuperscript{\cite{semicondutores2}}
		\end{small}		
	\end{figure}
	\pagebreak
	Com a variação de temperatura, alguns dos elétrons migram da banda de valência para a a banda de condução, devido ao ganho de energia no elétron.

	Em condições normais, os semicondutores possuem uma baixa condutividade, para contornar esse problema são dopadas impurezas. Essas impurezas podem ser do tipo doadora ou aceitadoras. O semicondutor dopado com uma impureza doadora é denominado do tipo N. Já os dopados com a impureza aceitadora é denominado do tipo P.\textsuperscript{\cite{semicondutores2}}

\subsubsection[Carregador de bateria]{Carregador de bateria}

\subsubsection[Conversores CC-CC]{Conversores CC-CC}

	Os conversores CC-CC (corrente contínua - corrente contínua) são circuitos eletrônicos que convertem as amplitudes de tensão e corrente contínua em outra amplitude. Essa conversão é realizada graças a semicondutores de potência operando como interruptores, além dos elementos passivos que formam o conjunto.\textsuperscript{\cite{semicondutores3}}
	
	Os principais tipos de conversores CC-CC são:
	
	\begin{minipage}{7cm}
		\begin{itemize}
			\item	\textit{Buck}
			\item	\textit{Boost}
			\item	\textit{Buck-boost}
			\item	Inversores
			\item	\textit{Forward}
			\item	\textit{Flyback}
	\end{itemize}
	\end{minipage}
	\begin{minipage}{7cm}
		\begin{itemize}
			\item	\textit{Push-pull}
			\item	\textit{Half bridge}
			\item	\textit{Full bridge}
			\item	\textit{Ćuk}
			\item	\textit{SEPIC}
		\end{itemize}
	\end{minipage}

	Para o projeto foram utilizados somente conversores dos tipos \textit{Buck} e \textit{Boost}. Esses dois tipos de conversores são conhecidos como conversores CC-CC diretos, pois neles existe a transferência direta da potência de entrada para a saída do conversor.
	
\subsubsection[Conversor Buck]{Conversor \textit{Buck}}

	O conversor \textit{Buck}, também conhecido como \textit{step-down}, é um conversor abaixador de tensão. Sua entrada é baseada em tensão e a saída em corrente, conforme pode ser visto na Figura \ref{Fig_buck}.\textsuperscript{\cite{inep}}
	
 	\begin{figure}[th]
		\caption{CONVERSOR \textit{BUCK}}
		\label{Fig_buck}
		\centering
		\includegraphics[width=0.5\linewidth]{./figs/buck}
			
		\begin{small}
			FONTE: INEP/EEL\textsuperscript{\cite{inep}}
		\end{small}		
	\end{figure}

	O principio de funcionamento do conversor é relativamente simples. Quando a chave (S) está em modo de condução, a corrente flui pelo indutor (L\textsubscript{o}) e pela saída do circuito. A tensão de entrada (V\textsubscript{i}) fornece a energia para a saída do circuito e para a magnetização do indutor (L\textsubscript{o}).
	
	No instante da abertura da chave (S) o diodo (D) entra em condução, a energia do indutor (L\textsubscript{o}) é transferida para a carga (R\textsubscript{o}), ou seja, o indutor (L\textsubscript{o}) é desmagnetizado.
	
	Esse tipo de conversor tem três possíveis modos de operação:
	
	\begin{itemize} 
		\item[\textbf{1º}] Condução contínua - sem anulaçao da corrente no indutor (L\textsubscript{o}) durante o período de comutação;
		\item[\textbf{2º}] Condução descontínua - com anulação da corrente no indutor (L\textsubscript{o}) a cada período de comutação;
		\item[\textbf{3º}] Condução crítica - a corrente no indutor (L\textsubscript{o}) fica no limiar de se anular a cada período de comutação.
	\end{itemize}
	
	O conversor \textit{Buck} tem a capacidade de diminuir a tensão na saída, mantendo a boa qualidade da corrente de saída. A corrente de entrada é descontínua, uma vez que o conversor precisa fazer o chaveamento para que o sistema possa funcionar perfeitamente.\textsuperscript{\cite{inep}}
	
\subsubsection[Conversor Boost]{Conversor \textit{Boost}}

	O conversor \textit{Boost}, também conhecido como \textit{step-up}, é um conversor elevador de tensão. Ao contrário do conversor \textit{Buck}, sua entrada é baseada em corrente e a saída em tensão, conforme pode ser visto na Figura \ref{Fig_boost}.\textsuperscript{\cite{inep}}

 	\begin{figure}[th]
		\caption{CONVERSOR \textit{BOOST}}
		\label{Fig_boost}
		\centering
		\includegraphics[width=0.5\linewidth]{./figs/boost}
			
		\begin{small}
			FONTE: INEP/EEL\textsuperscript{\cite{inep}}
		\end{small}		
	\end{figure}
	
	Seu princípio de funcionamento é, relativamente, similar ao do conversor \textit{Buck}. Quando a chave (S) está conduzindo, o indutor (L) é magnetizado, dessa forma a corrente flui somente entre a fonte (V\textsubscript{i}), o indutor (L) e a chave (S).
	
	Porém quando a chave (S) está aberta, o diodo (D) entra em condução, a fonte (V\textsubscript{i})e o indutor (L) fornecem energia à saída do circuito, aumentando a tensão na carga (R\textsubscript{o}).
	
	De forma resumida, o conversor \textit{Boost} pode apenas aumentar a tensão na saída do circuito, além de possuir uma boa qualidade da corrente de entrada. Porém possui uma corrente de saída descontínua, originado pelo chaveamento do que o circuito realiza para o seu funcionamento.\textsuperscript{\cite{inep}}
	
\subsubsection[Definição do carregador de bateria e dos conversores]{Definição do carregador de bateria e dos conversores}

	Inicialmente foram procurados diversos circuitos integrados de conversores capazes de realizar a elevação e diminuição da tensão de saída para a topologia inicial, onde existiria duas baterias em série com 4,2V. Porém após diversas horas de estudos não foram encontrados conversores \textit{Boost} capazes de realizar grandes elevações com alta eficiência. Somente os conversores com uma tensão de entrada superior aos 4,2V poderiam atender a especificação do projeto com a grande eficiência desejada.
	
	Dessa forma, foi alterada a topologia do projeto, onde a bateria de 4,2V, foram substituídas pela bateria de duas células em série de 7,4V, definida na Seção \ref{Sec_Bateria}.
	
	Durante as pesquisas, foi encontrado o projeto \textit{Open Source Satellite Initiative} e na \textit{wikipage} do projeto é disponibilizado um \textit{link} para uma \textit{data base} que contém dados de diversos testes de radiação realizados na \textit{NASA Goddard Space Flight Center}.\textsuperscript{\cite{OSSI}}\textsuperscript{\cite{OSSI2}}\textsuperscript{\cite{OSSI3}}
	
	Nessa \textit{data base} foram encontrados alguns conversores que poderiam ser utilizados no desenvolvimento do projeto, como por exemplo o \textit{Buck} TPS50601-SP da \textit{Texas Instruments}. Porém devido ao alto custo dos componentes presentes nessa lista o componente não foi uma solução viável ao projeto.
	
	Reiniciada a etapa de pesquisas, foi encontrado o \textit{WEBENCH Design Center}, que um software online da \textit{Texas Instruments}, no qual é possível inserir os parâmetros que deseja de um conversosr, como tensão de entrada e tensão de saída, e o software consulta sua \textit{data base} para encontrar o conversor conforme especificado e ainda reproduz o circuito de \textit{datasheet} da aplicação. Através do auxilio do \textit{WEBENCH Design Center}, foram identificados os seguintes conversores LM3488 (\textit{boost}) e o TPS563200 (\textit{buck}).
	
	Para a definição do conversor \textit{backup}, foi pesquisado um único conversor com diversas saídas, dessa forma apenas um circuito integrado seria capaz de devolver as tensões de 3,3V, 5V e 12V. A princípio foram encontrados algumas opções desses conversores com múltiplas saídas, porém todos os encontrados eram de 4 saídas. Após uma busca mais criteriosa, foi encontrado na \textit{Linear Technology} o LT1941 que possui as três saídas desejadas, sendo uma saída \textit{Boost} e duas saídas \textit{Buck}. Com a ajuda do \textit{LTspice IV}, software da própria \textit{Linear}, foi possível trabalhar em cima do \textit{application note} do \textit{datasheet} do componente para poder adaptá-lo as necessidades do projeto do Sistema de Gerenciamento de Energia para \textit{CubeSats}.
	
	


\subsection[Lista de componentes]{Lista de componentes}

	Seguindo as premissas apresentadas anteriormente, foi finalizada a lista de componentes necessários para o seguimento do projeto, também foi levado em consideração a questão financeira, portanto alguns componentes apresentados nessa lista podem ser substituídos após a realização de todos os ensaios e validações.
	
	Na Figura \ref{Fig_Lista}, é possível visualizar a lista de componentes eletrônicos definida pelo grupo.
	
	\begin{figure}[th]
		\caption{LISTA DE COMPONENTES ELETRÔNICOS}
		\label{Fig_Lista}
		\centering
		\includegraphics[width=1.0\linewidth]{./figs/comp}
			
		\begin{small}
			FONTE: Elaborado pelos autores.
		\end{small}		
	\end{figure}
	\pagebreak
	
	Foi solicitado um orçamento para a \textit{Farnell}, que devido a sua grande estrutura e interligação entre os estoques nacionais e internacionais, tinha a disposição praticamente todos os componentes necessários com exceção do carregador de bateria (BQ24005PWP), ficou R\$ 2.065,66, conforme pode ser visto no Anexo \ref{Orçamento}.
	


	
	
% ----------------------------------------------------------
% PROTÓTIPO 
% ----------------------------------------------------------
\chapter[Protótipo]{Protótipo}

\section[Ensaios possíveis]{Ensaios possíveis}
	
	Durante a fase de consolidação teórica foram identificados possíveis ensaios que poderiam ser feitos no sistema para validação do circuito como um todo. Esses ensaios são de essencial importância para o projeto, pois dessa forma podem ser identificados possíveis problemas que devem ser corrigidos antes do lançamento da missão.
	
\subsection[Validação dos subconjuntos]{Validação dos subconjuntos}

	Um dos ensaios identificados foi o de validação dos subconjuntos do Sistema de Gerenciamento de Energia para \textit{CubeSat}. O circuito do sistema como um todo foi dividido em vários subconjuntos de funções especificas para serem validados, ou seja, foi separada a parte do carregador de bateria, do conversor 3,3 V, do conversor de 5 V, do conversor de 12 V e do conversor \textit{backup}. Na Figura \ref{Fig_SubConj} pode ser visualizado duas placas com os subconjuntos do sistema.
	
	\begin{figure}[th]
		\caption{CIRCUITOS DOS SUBCONJUNTOS DO SISTEMA}
		\label{Fig_SubConj}
		\centering
		\includegraphics[width=0.7\linewidth]{./figs/placas_subconjuntos}
			
		\begin{small}
			FONTE: Foto tirada pelos autores.
		\end{small}		
	\end{figure}
	
\subsection[Ensaio de radiação]{Ensaio de radiação}	

	O ensaio de radiação consiste na emissão de radiação nos semicondutores a fim de se obter a assinatura da tensão e corrente após a exposição a radiação. Para esse ensaio foi necessário que os semicondutores estivessem decapados, ou seja, foi necessário que a dopagem de Sílicio do semicondutor estivesse exposta para que a emissão da radiação fosse direta no componente, fato que não ocorrerá no espaço, uma vez que a película do componente estará presente o que, por sua vez, acaba dando uma pequena filtrada na radiação. Esse teste foi realizado em conjunto com o Centro Universitário da Faculdade de Engenharia Industrial (FEI), com a Dr. Marcilei Guazzelli da FEI.
	
	Entre diversas pesquisas a respeito de possíveis testes de radiação, foi encontrado o projeto \textit{Open Source Satellite Initiative}. Na \textit{wikipage} do projeto \textit{OSSI} é disponibilizado um \textit{link} para um \textit{data base} que contém dados de testes de radiação da \textit{NASA Goddard Space Flight Center}.\textsuperscript{\cite{OSSI}}\textsuperscript{\cite{OSSI2}}\textsuperscript{\cite{OSSI3}}
	
	Nesse \textit{data base} foram encontrados dois componentes que constavam na lista de componentes da Figura \ref{Fig_Lista}, sendo eles o TMP36, produzido pela \textit{Analog Devices}, e o LMP2012, produzido pela \textit{Texas Instruments}. Foi realizada uma tentativa de transformação da lista de componentes inicial por uma contendo mais componentes presentes nesse \textit{data base}, porém, além do alto custo, o \textit{lead time} de muitos deles era elevadíssimos e poderia comprometer o desenvolvimento do projeto.
	
	O TMP36, que é um sensor de temperatura, foi testado para \textit{SEL}, pela \textit{NASA} em 2002, pelos engenheiros espaciais Christian Poivey e Hak Kim, chegando a conclusão que a sensibilidade \textit{SEL} do componente não era um fator preocupante para aplicações espaciais.\footnote{No Anexo \ref{TMP36} é possível realizar a verificação do relatório de teste do TMP36.\textsuperscript{\cite{TMP36}}}
	
	O LMP2012, trata-se de um amplificador operacional, foi testado para \textit{SEE}, \textit{SEL}, \textit{SET} e \textit{Laser}, os testes foram realizados pelos engenheiros espaciais da \textit{NASA}, Michael Campola e Anthony Phan no final de 2012, concluindo que existem algumas áreas que param de funcionar após uma indução de \textit{laser} pulsado.\footnote{No Anexo \ref{LMP2012} é possível realizar a verificação do relatório de teste do LMP2012.\textsuperscript{\cite{LMP2012}}}
	
	
\subsection[Ensaio térmico]{Ensaio térmico}

	O ensaio térmico foi realizado em parceria com o laboratório da USP, coordenado pelo Dr. Fábio Rodrigues. Esse ensaio consistia em uma câmara refrigerada que atingia uma temperatura próxima aos -80 ºC. 
	
	Após a validação dos subconjuntos e a obtenção das assinaturas de tensão e corrente desses circuitos, as placas foram levadas a USP, onde o Dr. Fabio Rodrigues, as colocou nessa câmara para que fossem obtidas as assinaturas de tensão e corrente nessa condição de baixas temperaturas, dessa forma foi possível validar, de forma aproximada, como esses subconjuntos se comportariam no ambiente espacial onde as temperaturas podem chegar próximas de -170 ºC.

\subsection[Ensaio de vácuo]{Ensaio de vácuo}	

	O ensaio térmico foi realizado em parceria com o laboratório da USP, coordenado pelo Dr. Fábio Rodrigues. Esse ensaio consistia em uma câmara de vácuo que atingia uma pressão de aproximadamente 10\textsuperscript{-9} mbar. 
	
	Após a validação e a obtenção das assinaturas de tensão e corrente dos ensaios anteriores, o Dr. Fabio Rodrigues colocou os circuitos na câmera de vácuo para que fossem obtidas as assinaturas de tensão e corrente nessa condição de pressões baixas podendo avaliar o comportamento dos circuitos com pressões próximas as que existem na órbita \textit{LEO}.
	
\subsection[Ensaio de termovácuo]{Ensaio de termovácuo}

	Por fim, após todas as validações anteriores os subconjuntos foram enviados para Vinhedo no Observatório Abrahão de Moraes onde o Dr. Douglas Galante, os colocou na câmera de termovácuo.
	
	\begin{figure}[th]
		\caption{CÂMERA DE TERMOCÁCUO}
		\label{Fig_SubConj}
		\centering
		\includegraphics[width=0.5\linewidth]{./figs/termovacuo}
			
		\begin{small}
			FONTE: Foto tirada pelos autores.
		\end{small}		
	\end{figure}
	
	Nesse ensaio é possível atingir baixas pressões (10\textsuperscript{-9} mbar) e também baixas temperaturas (X ºC), também é possível visualizar no \textit{LabView} a curva da pressão, sendo fácil constatar se algum problema ocorreu caso na curva tenha alguma subida repentina de pressão.
	
	\begin{figure}[th]
		\caption{CURVA DE PRESSÃO NO \textit{LABVIEW}}
		\label{Fig_SubConj}
		\centering
		\includegraphics[width=1.0\linewidth]{./figs/labview}
			
		\begin{small}
			FONTE: Foto tirada pelos autores.
		\end{small}		
	\end{figure}
	
\subsection[Avaliação dos ensaios]{Avaliação dos ensaios}

% ----------------------------------------------------------
% RESULTADOS E DISCUSSÕES
% ----------------------------------------------------------
%%%%%%%%%%%%%%%%%%%%%%%%%%%%%%%%%%%%%%%%%%%%%%%%%%%%%%%%%%%%%%\chapter[Resultados e discussões]{Resultados e discussões}

% ----------------------------------------------------------
% PLANO DE MARKETING
% ----------------------------------------------------------
%%%%%%%%%%%%%%%%%%%%%%%%%%%%%%%%%%%%%%%%%%%%%%%%%%%%%%%%%%%%%%\chapter[Plano de marketing]{Plano de marketing}

% ----------------------------------------------------------
% PLANO OPERACIONAL
% ----------------------------------------------------------
%%%%%%%%%%%%%%%%%%%%%%%%%%%%%%%%%%%%%%%%%%%%%%%%%%%%%%%%%%%%%%\chapter[Plano operacional]{Plano operacional}

% ----------------------------------------------------------
%PLANILHA FINANCEIRA 
% ----------------------------------------------------------
%%%%%%%%%%%%%%%%%%%%%%%%%%%%%%%%%%%%%%%%%%%%%%%%%%%%%%%%%%%%%%\chapter[Planilha financeira]{Planilha financeira}

% ----------------------------------------------------------
% CONCLUSÕES
% ----------------------------------------------------------
\chapter[Conclusões]{Conclusões} \label{ultimo_capitulo}


% ----------------------------------------------------------
% REFERÊNCIAS BIBLIOGRÁFICAS
% ----------------------------------------------------------
\bibliography{TCC_Sist_Gerenciamento_Potencia}

% ----------------------------------------------------------
% Anexos
% ----------------------------------------------------------

% ---
% Inicia os anexos
% ---
\begin{anexosenv}

% Imprime uma página indicando o início dos anexos
\partanexos

% ---
\chapter[Orçamento dos componentes da Farnell]{Orçamento dos componentes da \textit{Farnell}} \label{Orçamento}
% ---

	Orçamento enviado pelo Sr. Helder Sant'ana Viana da \textit{Farnell Newark element14}.

	\begin{figure}[th]
		\centering
		\includegraphics[width=0.85\linewidth]{./anexos/Pedido189845}	
	\end{figure}
	
	\begin{figure}[th]
		\centering
		\includegraphics[width=0.9\linewidth]{./anexos/Pedido1898452}	
	\end{figure}
	
	\begin{figure}[th]
		\centering
		\includegraphics[width=0.9\linewidth]{./anexos/Pedido1898453}	
	\end{figure}
	
\chapter[Relatório de teste do TMP36]{Relatório de teste do TMP36} \label{TMP36}
	
	Relatório de teste do TMP36, realizado em 08 de Agosto de 2002 pelos engenheiros Christian Poivey e Hak Kim.\textsuperscript{\cite{TMP36}}
	
	\begin{figure}[th]
		\centering
		\includegraphics[width=0.85\linewidth]{./anexos/tmp36_1}	
	\end{figure}	
	
	\begin{figure}[th]
		\centering
		\includegraphics[width=0.85\linewidth]{./anexos/tmp36_2}	
	\end{figure}	
	
	\begin{figure}[th]
		\centering
		\includegraphics[width=0.85\linewidth]{./anexos/tmp36_3}	
	\end{figure}	
	
	\begin{figure}[th]
		\centering
		\includegraphics[width=0.85\linewidth]{./anexos/tmp36_4}	
	\end{figure}			
	
\chapter[Relatório de teste do LMP2012]{Relatório de teste do LMP2012} \label{LMP2012}	
	
	Relatório de teste do LMP2012, realizado em 18 de Dezembro de 2012 pelos engenheiros Michael Campola e Anthony Phan.\textsuperscript{\cite{LMP2012}}
	
		\begin{figure}[th]
		\centering
		\includegraphics[width=0.85\linewidth]{./anexos/lmp2012_1}	
	\end{figure}	
	
	\begin{figure}[th]
		\centering
		\includegraphics[width=0.85\linewidth]{./anexos/lmp2012_2}	
	\end{figure}	
	
	\begin{figure}[th]
		\centering
		\includegraphics[width=0.85\linewidth]{./anexos/lmp2012_3}	
	\end{figure}	
	
	\begin{figure}[th]
		\centering
		\includegraphics[width=0.85\linewidth]{./anexos/lmp2012_4}	
	\end{figure}	
	
	
	
\end{anexosenv}



% ----------------------------------------------------------
% 
% ----------------------------------------------------------
%\chapter[]{}









%---------------------------------------------------------------------
% INDICE REMISSIVO
%---------------------------------------------------------------------
\phantompart
\printindex
%---------------------------------------------------------------------


\end{document}
